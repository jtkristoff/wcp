\documentclass[sigconf]{acmart}

% https://tex.stackexchange.com/questions/346292/how-to-remove-conference-information-from-the-acm-2017-sigconf-template
\settopmatter{printacmref=false} % Removes citation information below abstract
\renewcommand\footnotetextcopyrightpermission[1]{} % removes footnote with conference information in first column
\pagestyle{plain} % removes running headers

\bibliographystyle{ACM-Reference-Format}

\usepackage{graphicx}
\usepackage{listings}

\begin{document}

% use protected (~) whitespace to control title on second line
\title{The RPKI and Route Origin Validation: Advances~in~Deployment~and~Measurement}
\subtitle{PhD Qualifier Examination - Critical Review}
\author{John Kristoff}
\affiliation{\department{Computer Science}\institution{University of Illinois at Chicago}}
\email{jkrist3@uic.edu}

\begin{abstract}

Routing leaks and hijacks have plagued the Internet ecosystem since the
late 1990's.  These events, often unintentional, are recurring reminders
of the risk the Internet infrastructure is subject to.  The Resource
Public Key Infrastructure (RPKI) has arisen as an important and
necessary component to facilitate secure Internet routing.  Many network
operators are now publishing digitally signed certificates of address
prefixes and their associated route origins into the RPKI.  A smaller,
but burgeoning community of operators have begun deploying route origin
validation capabilities to limit the propagation of invalid routing
information in disagreement with the RPKI.  Now researchers are building
systems and techniques to evaluate how the RPKI is being populated and
used, drawing comparisons to other Internet infrastructure security
services such as the extensions to the domain name system.  Private and
localized validation policy decisions pose measurement challenges for
would-be secure routing surveyors however.  Operator polling, passive
observation of route updates, controlled route announcement experiments,
and data plane analysis are among the common techniques devised to
estimate deployment today.  When operators and route collection tools
incorporate route validation state in data collection systems,
researchers will be able to enhance their methods further still.

\end{abstract}

\maketitle

\section{Introduction}\label{sec:Introduction}

The Internet is a network of networks.  Each network that forms the
larger whole is commonly represented and referred to as an autonomous
system (AS).  To form a loop-free connected graph, an AS typically uses
the border gateway protocol (BGP), the Internet routing standard defined
by the Internet Engineering Task Force (IETF) request for comments (RFC)
4271 to connect to one or more other autonomous
systems.\cite{rekhter_border_2006}  AS-to-AS routers exchange and relay
reachability information using BGP update messages, which include one or
more address prefixes, an associated origin AS identifier, and other
attributes as appropriate.

BGP is unique among most common Internet standard routing protocols due
to the varied and widespread use of local policy configuration options
available at each router node and within an AS.  Each AS therefore may
form different views of the larger Internet topology by accepting,
rejecting, relaying, or altering routing information.  That is, while a
loop-free topology remains a critical feature of BGP, the shortest path
to any destination may be a secondary goal depending on local
administrator policy.  For example, one network may choose to reject or
withhold a route update containing a particular AS deemed untrustworty.
To illustrate another common scenario, an AS may devalue routing
information in order to prefer one path over another.  These types of
local policy decisions affect local path computations, but can also
impose policy onto neighboring networks and beyond, narrowing the path
selection decisions further downstream.

In addition to the added complexity local policy configurations imply,
the data conveyed in routing update messages between BGP speakers come
with no express means of authenticity.  This means that by default, as
long as the BGP message is well formed, a routing update with invalid
path data may find its way into the path computation process, wreaking
havoc and corrupting the computed topology of one or more autonomous
systems.  How does a BGP receiver evaluate routing information when an
autonomous neighbor is untrustworthy either due to accident or malicious
intent?  Approaches to protection from invalid routing data have varied,
haphazzadly dealt with in each autonomous system as a local policy
decision.  Consequently, overall Internet routing infrastructure
robustness against invalid routing data varies widely from AS to AS.
While a variety of solutions have been deployed, the Internet has lacked
a comprehensive, standard approach to authenticating routing data
distributed through BGP.  Unfortunately, the threats of invalid routing
data are not hypothetical.

In 1997, a unique set of circumstances coalesced at just the right place
and time to cause a major disruption in the Internet routing
system.\cite{barret_routing_1997}  A small Internet service provider
(ISP) both disaggregated and regenerated a large number of Internet
address prefixes back into the Internet, making it appear as if the
address prefixes in these routing updates all originated from the small
ISP.  These numerous and errorneous updates cascaded to many other
networks.  To many Internet router configurations the updates were
accepted and used to recompute paths.  Instead of traffic following the
best path during this event, many of intended destinations covered by
these updates became unreachable as routers started forwarding traffic
towards the wrong ISP.

Just over a decade later, another widely observed routing anomaly
impacted the popular YouTube streaming video
service.\cite{brown_pakistan_2008}  Pakistan Telecom, intending to block
access to YouTube from within its network accidentally leaked a specific
address prefix covering a portion of YouTube's address space in a route
update to an upstream provider.  This in turn propagated as a preferred
path to many other networks.  The proliferation of this update sent many
of the world's YouTube users towards Pakistan Telecom, which appeared in
the routing system as the best path, but effectively led to an Internet
version of a black hole.

The aforementioned network infrastructure mishaps, commonly referred to
as a \emph{hijack}, are among the most popular in the history of
Internet routing.  However, events such these, even when less well
known, recur with surprising frequency.  Often these events can be
classified as accidents, but there is evidence many anomalies have been
intentionally malicious.\cite{madory_bgp_2018}  Whether the events are
benign or costly, inadvertent or deliberate, mechanisms to limit their
occurrence have been an ongoing area of research, experimentation, and
development.

This paper evaluates the deployment and measurement efforts of a new
approach to hardening the Internet routing system against illegimate
routing updates.  The rest of the paper is organized as follows: In \S
\ref{sec:Background} the history of secure routing is divided into two
distinct eras.  The first era summarizes popular techniques and tools
network operators have traditionally deployed to limit the spread of
invalid or unwanted routing information.  The second era includes the
recent design of the resource public key infrastructure (RPKI), route
origin validation (ROV), and related standards efforts that form the
basis for the bulk of this paper.  In \S \ref{sec:Methodologies},
approaches to measuring the deployment of the RPKI and ROV from three
recent research papers that represent the state of the art are
summarized.  In \S \ref{sec:Discussion}, this paper offers fresh
perspective and criticism of the prior research under evaluation.  In \S
\ref{sec:Future Work}, an argument is made to include and leverage the
route validation state of ROV-enabled networks in route monitoring
systems to encourage deployment and enhance measurement.  The paper
concludes in \S \ref{sec:Conclusion}.

\section{Background}\label{sec:Background}

Secure routing conjures up images of different solutions for different
challenges.  To secure routing could imply mechanisms to thwart
byzantine failure or the use of encryption to protect the privacy of
routing data in transit.  It may may suggest a series of operator best
practices to limit access and configuration mistakes with hardware.  For
the purposes of this paper, and in terms of BGP, this paper is
particularly concerned with authenticating the origin AS in the AS path
attribute and the authorization granted the AS to originate an
associated address prefix in the update message.  Attention therefore
coverges on a subset of technologies that arose out of the Secure
Inter-Domain Routing (SIDR) working group in the IETF.

While a complete tutorial and history of approaches to secure routing on
the Internet is beyond the scope of this paper, a brief review of
popular approaches and technologies relevant to understanding the RPKI,
route origin authorizations (ROAs), and ROV is provided.  The reader is
encouraged to review the references provided for additional detail.

\subsection{Route Filters and Prefix Limits}

Two BGP capabilities frequently used to limit undesirable or unexpected
routing updates are route filters and prefix limits.  A route filter is
list of one or more rules that evaluates a received announcement in a
route update and decides whether to accept, reject, or alter the
announcement for local route computation.  A prefix limit is a
threshold, an expected upper limit of prefixes a BGP neighbor is
expected to announce to the receiving router.  A breach of a prefix
limit may trigger an alert or shut down the BGP peering session.

Network providers commonly use route filters to reject route
announcements they never expect to see.  For instance, if AS
\emph{49152} deems reachability to the IPv4 address prefix
\texttt{192.0.2.0/24} undesirable, a route filter could be applied to
reject any announcement for \texttt{192.0.2.0/24} or smaller overlapping
prefix from other networks.  See Listing~\ref{lst:route filter} for an
example BGP policy implementing a route filter to reject route
announcements for any address space covered by IETF RFC 1918
prefixes.\cite{moskowitz_address_1996}

\begin{lstlisting}[float,basicstyle=\footnotesize\ttfamily,caption={Example Junos route filter},label={lst:route filter}]
policy-options {
    prefix-list rfc1918 {
        10.0.0.0/8;                         
        172.16.0.0/12;                      
        192.168.0.0/16;                     
    }
    policy-statement sanitize-bgp {
        term rfc1918 {
            from {
                prefix-list-filter rfc1918 orlonger;
            }
            then reject;
        }
    }
}
\end{lstlisting}

Route filters can range from the relatively simple to thoroughly
complex.  Most router implementations provide an ability to peek into
almost all parts of the routing announcement and pass judgment in myriad
ways.  A creative operator may be able to prevent many invalid routing
updates using an elaborate set of route filters, but achieving complete
protection from all possible invalid routes would be a Herculean, if not
impossible task.  Nevertheless, route filters along with prefix limits
have been two of the most common and widely deployed capabilities to
limit invalid route propagtion on the Internet
today.\cite{durand_bgp_2015}

\subsection{Internet Routing Registries}

Since the early years of BGP deployment, the Internet community
identified the need for tools and a set of best practices to support the
sprawling routing infrastructure.  One of the early developments to
support these early challenges was the design and creation of Internet
Routing Registries (IRRs).\cite{bates_representation_1995}  The IRR
system was an outgrowth of the desire to express an autonomous system's
routing policy, primarily for troubleshooting and documentation
purposes.  Tools to help automate the validation of routing updates
against IRR data soon appeared.  In fact, when the \emph{AS 7007
Incident}\footnote{This is the name given to the routing event in 1997
described in this paper above.  7007 refers to the autonomous system
number assigned to the small ISP originating the invalid routes.}
occurred a number of operators suggested that the use of a routing
registry could have prevented the problem.

IRRs and IRR tools have been adopted by a number of networks, but the
data has been of varying completeness and
quality.\cite{khan_comparative_2013}  Perhaps the IRR system was ahead
of its time, or maybe it hasn't struck the right balance of cost,
consistency, ease of use, and utility for it to have become universally
deployed as a routing security mechanism.  It would take a number of
years and a few proposals before a system focused on solely on routing
security would take hold.

\subsection{S-BGP and soBGP}

At the turn of the 21st century, two extensions to BGP, S-BGP and soBGP,
were proposals to incorporate cryptographic authentication into the
routing system.\cite{kent_secure_2000}\cite{white_securing_2003}  Both
approaches specified a public key infrastructure (PKI) and both modified
BGP to incorporate authentication siginalling into the routing protocol.
Neither approach gained sufficient traction to become deployed systems
however.  Nonetheless, their use of a PKI foreshadowed a system to come
that would capture the imagination of operators and soon demonstrate
promising levels of real-world deployment.

\subsection{The RPKI}

At the center of modern secure routing technology today is the resource
public key infrastructure (RPKI), a distributed, hierarchical repository
containing cryptographically signed objects such as attestations of
autonomous systems associated with Internet address
prefixes.\cite{lepinski_infrastructure_2012}  The RPKI is formally
structured with each regional Internet registry (RIR) anchoring the
repository with the allocated address space they are responsible for.
Address holders work with their respective RIR to populate the RPKI or
to coordinate further delegation of the repository further.

The RPKI could be thought of as cross between the IRR system and a
PKI-based system like those of attempted by S-BGP and soBGP.  The RPKI,
like the IRRs, but unlike S-BGP and soBGP, is maintained outside of the
BGP protocol itself.  The RPKI leverages the fundamental principles of
an X.509-based PKI for routing data like those from S-BGP and soBGP, but
unlike the IRR, decrees a unified, hierarchical structure, anchored and
supported by all RIRs as an integrated service offering.  The initial
objects populating the RPKI provide the means by which to perform route
validation, an overview of which is covered below.

\subsection{ROAs}

Route origin authorizations (ROAs) are individually signed objects
associating allocated address prefixes with an authorized AS to
originate routes for these prefixes in the BGP system.  Neither BGP
route updates nor RIR address allocation records alone make this
relationship explicit.  With the RPKI system and the ability to populate
it with ROAs, the Internet routing system now has the two key building
blocks on the way toward a secure routing infrasturcture.

See Listing~\ref{lst:ROA} for an example ROA as recently published in
the American Registry for Internet Numbers (ARIN) RIR operational test
and evaluation environment.  A ROA can have only one origin AS
associated with one or more prefixes.  Each prefix can specify a
\emph{maxLength} value for the prefix, indicating the most specific
prefix length the origin AS is authorized to announce.  Like X.509
certificates, other attributes such as a digital signature and
expiration time are associated with each ROA.

\begin{lstlisting}[float,basicstyle=\footnotesize\ttfamily,caption={Example AS 20130 ROA},label={lst:ROA}]
ROA Name: DEPAUL AS20130
Origin AS: 20130
Validity Period: 02-12-2019 to 02-12-2029
Resources:
    2604:95C0::/32
    2620:0:2250::/48
    75.102.192.0/18
    216.220.176.0/20

-----BEGIN SIGNATURE-----
CphdY76ofLDDsBzKseuivh9fp8j8f95xZSQrs75MF+GU0nP5OKKtn
J6UvFLZH6L8YEWcxiGGuwTzgK0Puea+slXnXU+UgalmitqJOHwXbo
bAm7DCWou2wT2fIWqZHTUpX99/jFlSn34ozp2NFWJCT8ba4WlNgnI
sevnaeoe2KzEUbaawYCOskLU9B7aAPFhBHbuGGhQYpx08n3zLYj1R
MIOyOyl8NuSi3cfI0KbRZjhtIF3Pe9LebuqrwiBhRaFxzvFLM4g6z
Dff62/7Hnmt6PFio0Rn1UWPq2plDymT5peluCdDiL3M/DsGrEgqRf
wQKqll6HuRKaZVoHa0cNWPdw==
-----END SIGNATURE-----
\end{lstlisting}

\subsection{ROV}

Network operators can now create ROA objects for their address prefixes
and publish them in the RPKI.  This activity achieves only half of the
needed action to be useful, because the RPKI and ROAs are of little
value until they are used to validate BGP routing data.  To implement
ROV, network operators are expected to setup a set RPKI caches on the
local network, sometimes referred to as validators.  These validators
periodically fetch RPKI data from the designated anchors in repository
hierarchy.  The validators are then used in tandom with ROV-enabled
routers, whereby routers load validated ROA objects to be used as part
of the route processing path when BGP updates are received.  Most modern
routers support this functionality, allowing an operator to use ROV as
just another option to influence actions in the BGP policy processing
chain.

Colloquially known as route origin validation (ROV), BGP-speaking
routers can then use the RPKI and ROAs to influence processing rules
when BGP route updates are received.\cite{bush_origin_2014}  If a route
update is covered by a published ROA, a ROV-enabled router compares the
origin AS and prefix length in the update with an unexpired ROA.  The
action to take when ROV fails is a local policy decision, but most
operators either decrease the preference for the route or reject it
outright.\footnote{As of February 11, 2019, AT\&T began "dropping all
RPKI-invalid route announcements" from
peers\cite{borkenhagen_at&t/as7018_2019}  This was considered a major
milestone that could help encourage other providers yet to deploy
RPKI-based technology.}

\section{Methodologies}\label{sec:Methodologies}

Hmm.

\section{Discussion}\label{sec:Discussion}

Hah.

\section{Future Work}\label{sec:Future Work}

Baz.

\section{Conclusion}\label{sec:Conclusion}

Baf.

\begin{acks}

John is grateful for the networking position he holds at DePaul
University, which facilitates the ability to conduct meaningful
experiments such as those in ARIN's operational test and evaluation
environment (OTE) used to better understand the technology presented in
this paper.  John is also indebted to Patrick W.  Gilmore, who
generously made his time available to answer questions about the 1997
\emph{7007 Incident}.

\end{acks}

\bibliography{wcp}

\end{document}
