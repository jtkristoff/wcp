\documentclass[sigconf]{acmart}

% https://tex.stackexchange.com/questions/346292/how-to-remove-conference-information-from-the-acm-2017-sigconf-template
\settopmatter{printacmref=false} % Removes citation information below abstract
\renewcommand\footnotetextcopyrightpermission[1]{} % removes footnote with conference information in first column
\pagestyle{plain} % removes running headers

\bibliographystyle{unsrt}

\usepackage{graphicx}
\usepackage{listings}

\begin{document}

% use protected (~) whitespace to control title on second line
\title{The RPKI and Route Origin Validation: Advances~in~Deployment~and~Measurement}
\subtitle{PhD Qualifier Examination - Critical Review}
\author{John Kristoff}
\affiliation{\department{Computer Science}\institution{University of Illinois at Chicago}}
\email{jkrist3@uic.edu}

\begin{abstract}

Routing leaks and hijacks have plagued the Internet ecosystem since the
late 1990's.  These events, often unintentional, are recurring reminders
of the risk the Internet infrastructure is subject to.  The Resource
Public Key Infrastructure (RPKI) has arisen as an important and
necessary component to facilitate secure Internet routing.  Many network
operators are now publishing digitally signed certificates of address
prefixes and their associated route origins into the RPKI.  A smaller,
but burgeoning community of operators have begun deploying route origin
validation capabilities to limit the propagation of invalid routing
information in disagreement with the RPKI.  Now researchers are building
systems and techniques to evaluate how the RPKI is being populated and
used, drawing comparisons to other Internet infrastructure security
services such as the extensions to the domain name system.  Private and
localized validation policy decisions pose measurement challenges for
would-be secure routing surveyors however.  Operator polling, passive
observation of route updates, controlled route announcement experiments,
and data plane analysis are among the common techniques devised to
estimate deployment today.  When operators and route collection tools
incorporate route validation state in data collection systems,
researchers will be able to enhance their methods further still.

\end{abstract}

\maketitle

\section{Introduction}\label{sec:Introduction}

The Internet is network of networks.  Each network that forms the larger
whole is commonly represented and referred to as an autonomous system
(AS).  To form a loop-free connected graph, an AS typically uses the
border gateway protocol (BGP), the Internet routing standard defined in
IETF RFC 4271 to connect to one or more other autonomous
systems.\cite{rekhter_border_2006}  AS-to-AS routers exchange and relay
reachability information using BGP update messages, which include one or
more address prefixes, an associated origin AS identifier, and other
attributes as appropriate.

BGP is unique among most common Internet standard routing protocols due
to the varied and widespread use of local policy configuration options
available at each router node and within an AS.  Each AS therefore may
form different views of the larger Internet topology by accepting,
rejecting, relaying, or altering routing information.  That is, while a
loop-free topology remains a critical feature of BGP, the shortest path
to any destination may be a secondary goal depending on local
administrator policy.  For example, one network may choose to reject or
withhold a route update containing a particular AS deemed untrustworty.
To illustrate another common scenario, an AS may devalue routing
information in order to prefer one path over another.  These types of
local policy decisions affect local path computations, but can also
impose policy onto neighboring networks and beyond, narrowing the path
selection decisions further downstream.

In addition to the added complexity local policy configurations imply,
the data conveyed in routing update messages between BGP speakers come
with no express means of authenticity.  This means that by default, as
long as the BGP message is well formed, a routing update with invalid
path data may find its way into the path computation process, wreaking
havoc and corrupting the computed topology of one or more autonomous
systems.  How does a BGP receiver evaluate routing information when an
autonomous neighbor is untrustworthy either due to accident or malicious
intent?  Approaches to protection from invalid routing data have varied,
again dealt with in each autonomous system as a local policy decision.
Consequently, overall Internet routing infrastructure robustness against
invalid routing data varies widely from AS to AS.  While a variety of
solutions have been deployed, the Internet has lacked a comprehensive,
standard approach to authenticating routing data distributed through
BGP.  Unfortunately, the threats of invalid routing data are not
hypothetical.

In 1997, a unique set of circumstances coalesced at just the right place
and time to cause a major disruption in the Internet routing
system.\cite{barret_routing_1997}  A small Internet service provider
(ISP) both disaggregated and redistributed a large number of Internet
address prefixes, sending these updates upstream to a larger ISP.  These
numerous and errorneous updates cascaded to many other networks.  To
many Internet router configurations the updates were accepted and used
to recompute paths.  Instead of of a better path, many of intended
destinations covered by these updates became unreachable as routers
started forwarding traffic towards the wrong ISP.

Just over a decade later, another widely observed routing anomaly
impacted the popular YouTube streaming video
service.\cite{brown_pakistan_2008}  Pakistan Telecom, intending to block
access to specific addresses used by YouTube from within its network
accidentally leaked a route update to an upstream provider, which in
turn propagated this preferred route update to many other networks.
The proliferation of this update sent many of the world's YouTube users
towards Pakistan Telecom, effectively into an Internet black hole.

The aforementioned network infrastructure mishaps are among the most
popular in the history of Internet routing.  However, events such these,
even when less well known, recur with surprising frequency.  Often these
events can be classified as accidents, but there is evidence many
anomalies have been intentionally malicious.\cite{madory_bgp_2018}
Whether the events are benign or costly, inadvertent or deliberate,
mechanisms to limit their occurrence have been an ongoing area of
research, experimentation, and development.

This paper evaluates the deployment and measurement efforts of a new
approach to hardening the Internet routing system against illegimate
routing updates.  The rest of the paper is organized as follows: In \S
\ref{sec:Background} the history of secure routing is considered as two
distinct eras.  The first era summarizes popular techniques and tools
network operators have deployed to limit the spread of invalid or
unwanted routing information.  The second era presents an overview of
the resource public key infrastructure (RPKI), route origin validation
(ROV), and related standards efforts that form the basis for the bulk of
this paper.  In \S \ref{sec:Methodologies}, approaches to measuring the
deploying of the RPKI and ROV from three recent research papers are
summarized.    \S \ref{sec:Discussion} further evaluates and critiques
the prior research.  \S \ref{sec:Future Work} argues for the inclusion
and utilization of route validation states in route monitoring systems
of the future.  The paper concludes in \S \ref{sec:Conclusion}.

\section{Background}\label{sec:Background}

Addresses/Prefixes.  ASNs.  RIRs.  Routing.  Routing security.  History
and time line to RPKI.

The Resource Public Key
Infrastructure (RPKI), an Internet routing-specific PKI, anchored at the
five regional Internet registries (RIRs) is the starting point for a
repository of cryptographically signed route object authorizations
(ROAs).  ROAs

\section{Methodologies}\label{sec:Methodologies}

Hmm.

\section{Discussion}\label{sec:Discussion}

Hah.

\section{Future Work}\label{sec:Future Work}

Baz.

\section{Conclusion}\label{sec:Conclusion}

Baf.

\bibliography{wcp}

\end{document}
