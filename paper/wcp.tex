\documentclass[sigconf]{acmart}

% https://tex.stackexchange.com/questions/346292/how-to-remove-conference-information-from-the-acm-2017-sigconf-template
\settopmatter{printacmref=false} % Removes citation information below abstract
\renewcommand\footnotetextcopyrightpermission[1]{} % removes footnote with conference information in first column
\pagestyle{plain} % removes running headers

\bibliographystyle{ACM-Reference-Format}

\usepackage{graphicx}
\usepackage{listings}

\begin{document}

% use protected (~) whitespace to control title on second line
\title{The RPKI and Route Origin Validation: Advances~in~Deployment~and~Measurement}
\subtitle{PhD Qualifier Examination - Critical Review}
\author{John Kristoff}
\affiliation{\department{Computer Science}\institution{University of Illinois at Chicago}}
\email{jkrist3@uic.edu}

\begin{abstract}

Routing leaks and hijacks have plagued the Internet ecosystem since the
late 1990's.  These events, often unintentional, are recurring reminders
of the risk the Internet infrastructure is subject to.  The Resource
Public Key Infrastructure (RPKI) has arisen as an important and
necessary component to facilitate secure Internet routing.  Many network
operators are now publishing digitally signed certificates of address
prefixes and their associated route origins into the RPKI.  A smaller,
but burgeoning community of operators have begun deploying route origin
validation capabilities to limit the propagation of invalid routing
information in disagreement with the RPKI.  Now researchers are building
systems and techniques to evaluate how the RPKI is being populated and
used, drawing comparisons to other Internet infrastructure security
services such as the extensions to the domain name system.  Private and
localized validation policy decisions pose measurement challenges for
would-be secure routing surveyors however.  Operator polling, passive
observation of route updates, controlled route announcement experiments,
and data plane analysis are among the common techniques devised to
estimate deployment today.  When operators and route collection tools
incorporate route validation state in data collection systems,
researchers will be able to enhance their methods further still.

\end{abstract}

\maketitle

\section{Introduction}\label{sec:Introduction}

The Internet is a network of networks.  Each network that forms the
larger whole is commonly represented and referred to as an autonomous
system (AS).  To form a loop-free connected graph, an AS typically uses
the border gateway protocol (BGP), the Internet routing standard defined
by the Internet Engineering Task Force (IETF) request for comments (RFC)
4271 to connect to one or more other autonomous
systems.\cite{rekhter_border_2006}  AS-to-AS routers exchange and relay
reachability information using BGP update messages, which include one or
more address prefixes, an associated origin AS identifier, and other
attributes as appropriate.

BGP is unique among most common Internet standard routing protocols due
to the varied and widespread use of local policy configuration options
available at each router node and within an AS.  Each AS therefore may
form different views of the larger Internet topology by accepting,
rejecting, relaying, or altering routing information.  That is, while a
loop-free topology remains a critical feature of BGP, the shortest path
to any destination may be a secondary goal depending on local
administrator policy.  For example, one network may choose to reject or
withhold a route update containing a particular AS deemed untrustworthy.
To illustrate another common scenario, an AS may devalue routing
information in order to prefer one path over another.  These types of
local policy decisions affect local path computations, but can also
impose policy onto neighboring networks and beyond, narrowing the path
selection decisions further downstream.

In addition to the added complexity local policy configurations imply,
the data conveyed in routing update messages between BGP speakers come
with no express means of authenticity.  This means that by default, as
long as the BGP message is well formed, a routing update with invalid
path data may find its way into the path computation process, wreaking
havoc and corrupting the computed topology of one or more autonomous
systems.  How does a BGP receiver evaluate routing information when an
autonomous neighbor is untrustworthy either due to accident or malicious
intent?  Approaches to protection from invalid routing data have varied,
haphazardly dealt with in each autonomous system as a local policy
decision.  Consequently, overall Internet routing infrastructure
robustness against invalid routing data varies widely from AS to AS.
While a variety of solutions have been deployed, the Internet has lacked
a comprehensive, standard approach to authenticating routing data
distributed through BGP.  Unfortunately, the threats of invalid routing
data are not hypothetical.

In 1997, a unique set of circumstances coalesced at just the right place
and time to cause a major disruption in the Internet routing
system.\cite{barret_routing_1997}  A small Internet service provider
(ISP) both disaggregated and regenerated a large number of Internet
address prefixes back into the Internet, making it appear as if the
address prefixes in these routing updates all originated from the small
ISP.  These numerous and erroneous updates cascaded to many other
networks.  To many Internet router configurations the updates were
accepted and used to recompute paths.  Instead of traffic following the
best path during this event, many of intended destinations covered by
these updates became unreachable as routers started forwarding traffic
towards the wrong ISP.

Just over a decade later, another widely observed routing anomaly
impacted the popular YouTube streaming video
service.\cite{brown_pakistan_2008}  Pakistan Telecom, intending to block
access to YouTube from within its network accidentally leaked a specific
address prefix covering a portion of YouTube's address space in a route
update to an upstream provider.  This in turn propagated as a preferred
path to many other networks.  The proliferation of this update sent many
of the world's YouTube users towards Pakistan Telecom, which appeared in
the routing system as the best path, but effectively led to an Internet
version of a black hole.

The aforementioned network infrastructure mishaps, commonly referred to
as a \emph{hijack}, are among the most popular in the history of
Internet routing.  However, events such these, even when less well
known, recur with surprising frequency.  Often these events can be
classified as accidents, but there is evidence many anomalies have been
intentionally malicious.\cite{madory_bgp_2018}  Whether the events are
benign or costly, inadvertent or deliberate, mechanisms to limit their
occurrence have been an ongoing area of research, experimentation, and
development.

This paper evaluates the deployment and measurement efforts of a new
approach to hardening the Internet routing system against illegitimate
routing updates.  The rest of the paper is organized as follows: In \S
\ref{sec:Background} the history of secure routing is divided into two
distinct eras.  The first era summarizes popular techniques and tools
network operators have traditionally deployed to limit the spread of
invalid or unwanted routing information.  The second era includes the
recent design of the resource public key infrastructure (RPKI), route
origin validation (ROV), and related standards efforts that form the
basis for the bulk of this paper.  In \S \ref{sec:Methodologies},
approaches to measuring the deployment of the RPKI and ROV from three
recent research papers that represent the state of the art are
summarized.  In \S \ref{sec:Discussion}, this paper offers fresh
perspective and criticism of the prior research under evaluation.  In \S
\ref{sec:Future Work}, an argument is made to include and leverage the
route validation state of ROV-enabled networks in route monitoring
systems to encourage deployment and enhance measurement.  The paper
concludes in \S \ref{sec:Conclusion}.

\section{Background}\label{sec:Background}

Secure routing conjures up images of different solutions for different
challenges.  One might dream of mechanisms to thwart byzantine failure
or the use of encryption to protect the privacy of routing data in
transit.  It may may suggest a series of access control best practices
or sane configuration defaults on hardware.  For the purposes of this
paper, and in terms of BGP, this paper is particularly concerned with
authenticating the origin AS in the AS path attribute and the
authorization granted the AS to originate an associated address prefix
in the update message.  Attention therefore converges on a subset of
technologies that arose out of the Secure Inter-Domain Routing (SIDR)
working group in the IETF.

While a complete tutorial and history of approaches to secure routing on
the Internet is beyond the scope of this paper, a brief review of
popular approaches and technologies relevant to understanding the RPKI,
route origin authorizations (ROAs), and ROV is provided.  The reader is
encouraged to review the references provided for additional detail.

\subsection{Route Filters and Prefix Limits}

Two BGP capabilities frequently used to limit undesirable or unexpected
routing updates are route filters and prefix limits.  A route filter is
list of one or more rules that evaluates a received announcement in a
route update and decides whether to accept, reject, or alter the
announcement for local route computation.  A prefix limit is a
threshold, an expected upper limit of prefixes a BGP neighbor is
expected to announce to the receiving router.  A breach of a prefix
limit may trigger an alert or shut down the BGP peering session.

Network providers commonly use route filters to reject route
announcements they never expect to see.  For instance, if AS
\emph{49152} deems reachability to the IPv4 address prefix
\texttt{192.0.2.0/24} undesirable, a route filter could be applied to
reject any announcement for \texttt{192.0.2.0/24} or smaller overlapping
prefix from other networks.  See Listing~\ref{lst:route filter} for an
example BGP policy implementing a route filter to reject route
announcements for any address space covered by IETF RFC 1918
prefixes.\cite{moskowitz_address_1996}

\begin{lstlisting}[float,basicstyle=\footnotesize\ttfamily,caption={Example Junos route filter},label={lst:route filter}]
policy-options {
    prefix-list rfc1918 {
        10.0.0.0/8;                         
        172.16.0.0/12;                      
        192.168.0.0/16;                     
    }
    policy-statement sanitize-bgp {
        term rfc1918 {
            from {
                prefix-list-filter rfc1918 orlonger;
            }
            then reject;
        }
    }
}
\end{lstlisting}

Route filters can range from the relatively simple to thoroughly
complex.  Most router implementations provide an ability to peek into
almost all parts of the routing announcement and pass judgment in myriad
ways.  A creative operator may be able to prevent many invalid routing
updates using an elaborate set of route filters, but achieving complete
protection from all possible invalid routes would be a Herculean, if not
impossible task.  Nevertheless, route filters along with prefix limits
have been two of the most common and widely deployed capabilities to
limit invalid route propagation on the Internet
today.\cite{durand_bgp_2015}

\subsection{Internet Routing Registries}

Since the early years of BGP deployment, the Internet community
identified the need for tools and a set of best practices to support the
sprawling routing infrastructure.  One of the early developments to
support these early challenges was the design and creation of Internet
Routing Registries (IRRs).\cite{bates_representation_1995}  The IRR
system was an outgrowth of the desire to express an autonomous system's
routing policy, primarily for troubleshooting and documentation
purposes.  Tools to help automate the validation of routing updates
against IRR data soon appeared.  In fact, when the \emph{AS 7007
Incident}\footnote{This is the name given to the routing event in 1997
described in this paper above.  7007 refers to the autonomous system
number assigned to the small ISP originating the invalid routes.}
occurred a number of operators suggested that the use of a routing
registry could have prevented the problem.

IRRs and IRR tools have been adopted by a number of networks, but the
data has been of varying completeness and
quality.\cite{khan_comparative_2013}  Perhaps the IRR system was ahead
of its time, or maybe it hasn't struck the right balance of cost,
consistency, ease of use, and utility for it to have become universally
deployed as a routing security mechanism.  It would take a number of
years and a few proposals before a system focused on solely on routing
security would take hold.

\subsection{S-BGP and soBGP}

At the turn of the 21st century, two extensions to BGP, S-BGP and soBGP,
were proposals to incorporate cryptographic authentication into the
routing system.\cite{kent_secure_2000}\cite{white_securing_2003}  Both
approaches specified a public key infrastructure (PKI) and both modified
BGP to incorporate authentication signaling into the routing protocol.
Neither approach gained sufficient traction to become deployed systems
however.  Nonetheless, their use of a PKI foreshadowed a system to come
that would capture the imagination of operators and soon demonstrate
promising levels of real-world deployment.

\subsection{The RPKI}

At the center of modern secure routing technology today is the resource
public key infrastructure (RPKI), a distributed, hierarchical repository
containing cryptographically signed objects such as attestations of
autonomous systems associated with Internet address
prefixes.\cite{lepinski_infrastructure_2012}  The RPKI is formally
structured with each regional Internet registry (RIR) anchoring the
repository with the allocated address space they are responsible for.
Address holders work with their respective RIR to populate the RPKI or
to coordinate a delegation of partial repository data further.

The RPKI could be thought of as cross between the IRR system and a
PKI-based system like those of attempted by S-BGP and soBGP.  The RPKI,
like the IRRs, but unlike S-BGP and soBGP, is maintained outside of the
BGP protocol itself.  The RPKI leverages the fundamental principles of
an X.509-based PKI for routing data like those from S-BGP and soBGP, but
unlike the IRR, decrees a unified, hierarchical structure, anchored and
supported by all RIRs as an integrated service offering.  The initial
objects populating the RPKI provide the means by which to perform route
validation, an overview of which is covered below.

\subsection{ROAs}

Route origin authorizations (ROAs) are individually signed objects
associating allocated address prefixes with an AS authorized to
originate routes for these prefixes in the BGP system.  Neither BGP
route updates nor RIR address allocation records alone make this
relationship explicit.  With the RPKI system and the ability to populate
it with ROAs, the Internet routing system now has the two key building
blocks on the way toward a secure routing infrastructure.

A ROA can have only one origin AS associated with one or more prefixes.
Each prefix can specify a \emph{maxLength} value for the prefix,
indicating the most specific prefix length the origin AS is authorized
to announce.  Like X.509 certificates, other attributes such as a
digital signature and expiration time are associated with each ROA.  See
Listing~\ref{lst:ROA} for an example ROA as recently published in the
American Registry for Internet Numbers (ARIN) RIR operational test and
evaluation environment.

\begin{lstlisting}[float,basicstyle=\footnotesize\ttfamily,caption={Example AS 20130 ROA},label={lst:ROA}]
ROA Name: DEPAUL AS20130
Origin AS: 20130
Validity Period: 02-12-2019 to 02-12-2029
Resources:
    2604:95C0::/32
    2620:0:2250::/48
    75.102.192.0/18
    216.220.176.0/20

-----BEGIN SIGNATURE-----
CphdY76ofLDDsBzKseuivh9fp8j8f95xZSQrs75MF+GU0nP5OKKtn
J6UvFLZH6L8YEWcxiGGuwTzgK0Puea+slXnXU+UgalmitqJOHwXbo
bAm7DCWou2wT2fIWqZHTUpX99/jFlSn34ozp2NFWJCT8ba4WlNgnI
sevnaeoe2KzEUbaawYCOskLU9B7aAPFhBHbuGGhQYpx08n3zLYj1R
MIOyOyl8NuSi3cfI0KbRZjhtIF3Pe9LebuqrwiBhRaFxzvFLM4g6z
Dff62/7Hnmt6PFio0Rn1UWPq2plDymT5peluCdDiL3M/DsGrEgqRf
wQKqll6HuRKaZVoHa0cNWPdw==
-----END SIGNATURE-----
\end{lstlisting}

\subsection{ROV}

Colloquially known as route origin validation (ROV), BGP-speaking
routers can use the RPKI and ROAs to influence routing policy decisions
when BGP route updates are received.\cite{bush_origin_2014}  If a route
update is covered by a published, valid ROA, a ROV-enabled router
compares the origin AS and prefix length in the update with
corresponding data found in the ROA.  What a router will do when a route
update experiences a ROV failure is a local policy decision, but most
operators either decrease the preference for the route or reject it
outright.\footnote{As of February 11, 2019, AT\&T began "dropping all
RPKI-invalid route announcements" from
peers\cite{borkenhagen_at&t/as7018_2019}  This was considered a major
milestone that could help encourage other providers yet to deploy
RPKI-based technology.}

To implement ROV, network operators are expected to deploy a set of RPKI
caches, sometimes referred to as validators, on their local network.
These validators periodically fetch repository data from the designated
anchors in RPKI hierarchy.  The validators are then used in tandem with
ROV-enabled routers, whereby routers load validated ROA object detail,
suitable for use as part of the BGP policy processing rules setup by the
local network operator.

The RPKI, ROA management, and ROV deployment has stirred the interest of
standards bodies, networks operators, and vendors in recent years.
Consequently, measuring the deployment of the RPKI and ROV based on
published ROAs has attracted an equal amount of attention from the
Internet research community.  The next section will critically analyze
some of this most recent research activity.

\section{Methodologies}\label{sec:Methodologies}

The RPKI infrastructure along with associated ROV capable validators
and routers are still relatively new technologies.  The base
specifications were published in 2012.  Deployment began in earnest
shortly thereafter and by 2013 public monitoring emerged to monitor
deployment and collect statistics.\cite{nist_rpki_2019}  An early
research paper focused on identifying how the newly deployed RPKI was
being used, paying particular attention to the quality of ROAs network
operators created for their prefixes.\cite{iamartino_measuring_2015}
This work was among the first to consider whether routing data would
validate against published ROAs.  The authors performed an
\emph{offline} passive analysis, comparing a snapshot of published ROAs
with a snapshot of routing data from the Route Views
project.\cite{oregon_routeviews_2019}  Ever since, research interest in
the deployment of the RPKI and ROV have begun to take shape.  More
deployment strategies and measurement studies are now appearing as
common themes in networking and security venues.  This section explores
some recent advances in deployment and measurement.

\subsection{Uncontrolled Passive ROV Measurement}

RPKI repository data, and specifically the ROAs, can be freely and
easily retrieved.  Normally the ROAs are ultimately fed to the routing
system for ROV, but can be used to inform RPKI usage and measurement
studies.  Coupled with route collector systems such as those provided by
the RouteViews project, researchers can study real-world data without
having direct access to Internet routers.  Inferences can be made from
this passively collected data, but a hypothesis cannot be tested without
a control variable present.

\subsubsection{Method}

Address allocations and ROAs are published in an open and transparent
repository, a registry or RPKI respectively.  ROV however is a local AS
policy decision, unpublished in any central repository and generally
unknowable to those unaffiliated with the local network administration.
The authors in \cite{gilad_are_2017} propose a method to uncover ROV
enforcement in the Internet by evaluating routes received at RouteViews
sensors or vantage points.  This study searches a sensor's collected
data for an AS originating both valid and invalid prefixes when covering
ROAs are published.  The supposition is when both valid and invalid
prefixes arrive at a sensor from a common origin AS, ROV is not enforced
along the AS path.

Figure \ref{fig:rov1} demonstrates this inference approach.  Two sensors
see slightly different views of the topology.  AS 64500 originates two
prefixes, one valid, another invalid when evaluated with the RPKI.  If
AS 64501 is ROV enforcing, only the valid route will be seen by sensor
1.  Conversely, if AS 64502 is not ROV enforcing, sensor 2 will see both
valid and invalid prefixes.

\begin{figure}
  \centering
    \includegraphics[width=0.4\textwidth]{rov1}
  \caption{Passive ROV Detection}
  \label{fig:rov1}
\end{figure}

\subsubsection{Evaluation}

In response to \cite{gilad_are_2017} another group of researchers found
a number of limitations with this approach.\cite{reuter_towards_2018}
One set of problems stem from the nuances of local BGP policy decisions
made by each autonomous system.  For example, an AS that does not
enforce ROV may receive and process competing invalid and valid routes,
but may select the valid route as the best path for reasons unrelated to
ROV.  Only the best route will be propagated further, fooling an
uncontrolled passive measurement approach.  Another problem with this
approach is that it requires the origination of an invalid route to
detect whether or not an AS along the path is ROV enforcing.  An AS
originating both valid and invalid prefixes may be a relatively rare
event.

The authors in \cite{gilad_are_2017} claimed to have discovered nine of
the top ISPs were enforcing ROV.  However, \cite{reuter_towards_2018}
could not duplicate these results.  The response paper attempted to
recreate the measurement experiment and found the technique exhibited
discrepancies dependent upon the RouteViews sensor data being
evaluated.  The original paper did not make their methods or specific
data sources available so it was impossible to reconcile the differences
seen between the two results.

Publishing ROAs, at least for address space that is not sub-allocated to
downstream networks, is a relatively simple operation compared to ROV
enforcement.  It would stand to reason that an AS enforcing ROV is
likely to also be publishing some ROAs, but the converse seems unlikely.
Neither papers took into account whether any of the networks considered
ROV-enforcing also had published ROAs.  Only
the\cite{reuter_towards_2018} paper disclosed a list of autonomous
systems they detected to be enforcing ROV.  Historical ROAs are not
readily available, but only two of the four networks identified as
ROV-enforcing using a passive uncontrolled measurement approach have
published ROAs as of this writing.  With a controlled active
measurement approach, detailed in the next section, the three networks
identified as ROV-enforcing all currently publish ROAs.

\subsection{Controlled Active ROV Measurement}

Rather than rely on uncontrolled passive ROV measurement data derived
from networks whose connectivity and BGP policy are largely hidden from
view, a controlled experiment with active measurement could be
undertaken.  To measure ROV using this approach
\cite{reuter_towards_2018} proposes a series of experiments that start
with a set of prefixes and ROAs under their supervision.  Devising a
series of experiments with these prefixes can produce direct
observations that can in turn infer causation, leading to more reliable
results.

\subsubsection{Method}

To observe ROV enforcement in the wild, researchers using this technique
leveraged the PEERING testbed to originate two IPv4 /24 prefixes into
the Internet BGP routing system.\cite{schlinker_peering:_2014}  Of the
two prefixes announced, one is a control prefix whose announcement will
always pass validation with a corresponding ROA.  The other prefix
oscillates between a valid and invalid state when its associated ROA
changes during the experiment.

To minimize ambiguity about a network's ROV policy, only networks with
vantage points (referred to as sensors by other researchers) and direct
neighbors of the PEERING testbed are evaluated.  Furthermore, each
vantage point is verified at the start of the experiment to ensure it
can see both prefixes with valid associated ROAs.

If a vantage point is unable to see the invalid experimental prefix or
has a different route for the invalid prefix, the neighbor at this
vantage is presumed ROV-enforcing.

\subsubsection{Evaluation}

This approach significantly improves the reliability of detecting ROV
enforcement compared to the uncontrolled passive approach.  However,
this technique is limited in the number of networks it can evaluate
since the PEERING testbed connects to a relatively small number of
neighbors. The number of top ISPs connecting to the testbed is roughly
15\% of the top 100.  Nonetheless, the controlled active approach
greatly improves the accuracy and robustness over an uncontrolled
passive approach.  The team from \cite{reuter_towards_2018} identified
three networks performing ROV using this method, all three confirmed
when this result when contacted.  Conceivably this approach conducted on
a larger scale, using larger well-connected networks instead of just the
testbed, could uncover a high number of ROV-enforcing networks.

This method might be susceptible to false positives if ROA propagation
time, router recalculation behavior, or validator caching intervals are
not considered.  These researchers understood these potential problems
and designed experiments to limit their effects by withdrawing a route
before a ROA change and then subsequently issuing an announcement once
the change was submitted.  A potentially useful follow up study would
seek to characterize ROA propagation behavior in the wild.

\subsection{Data Plane Experiments}

Thus far, approaches to ascertain ROV enforcement have centered on
routing messages or routing table snapshots.  These sources of insight
derive from the network control plane.  That is, this data collected
says something about how the network is organized and the decisions
individual routers purport to make, but taken in isolation does not say
how data traffic actually traverses real world networks on an end-to-end
basis.  Furthermore, unless it were possible to compile a complete
picture of the entire Internet routing system, there will likely be some
disparity in the conclusions.  Posed with these challenges, evaluation
of ROV with the help of data traffic experiments, or the data plane, may
enhance prior methods.

\subsubsection{Method}

The authors of \cite{reuter_towards_2018} suggest in passing a way to
extend the reach of their controlled active method when a vantage point
is not available on a network peered with the testbed, but traceroute,
reverse traceroute, or a RIPE Atlas probe
is.\cite{katz-bassett_reverse_2010}\cite{ripe_ncc_staff_ripe_2015} Data
traffic would be sent from a candidate ROV network connected to the
testbed network, towards a monitor on the origin AS.  Observing the
ingress peer interface for which trace traffic arrives, ROV enforcement
could be observed.

In \cite{hlavacek_practical_2018}, the authors announce a pair of
prefixes both originating from two distinct autonomous systems.  A pair
of ROAs for these prefixes are created.  However, while each AS
announces both prefixes, only one AS is valid for each prefix according
to published ROAs at a time.  Without ROV, it would appear the two
prefixes are each multihomed to each AS or used in a common BGP anycast
arrangement.  For this experiment, traceroute probes towards the
prefixes from RIPE Atlas probes are run.  Each router hop from the
traceroute output is translated to the associated AS originating a
prefix for that router hop address.  This in essence constructs a
virtual data plane derived AS path.  Inference for ROV enforcement can
then made based on the networks seen or not seen in the traceroute data
plane paths for a particular test prefix and destination AS.

The same authors devise a measurement approach where observation and
control are available at only origin AS end of the path.   Here they
evaluate SYN/ACK responses from TCP destination port 80 probes from test
prefixes to Alexa top websites.\cite{noauthor_alexa_2018}  This method
makes inferences about ROV enforcement when responses arrive at the
invalid AS originating an associated prefix.

\subsubsection{Evaluation}

Using data plane inference to detect ROV is problematic.  If the source
and destination networks are peering neighbors, results can be reliable,
but otherwise autonomous BGP policy decisions and can mask a number of
unforeseen reasons a router or network selects a best path.  The authors
of \cite{hlavacek_practical_2018} acknowledge the uncertainty of their
results and notably claim to prove only 0.1\% of evaluated paths are ROV
enforcing.  The authors did not disclose the networks detected as ROV
enforcing nor an indication whether administrators verified their
findings.  It is reasonable to retain some amount of skepticism without
details of the findings, raw data, or manual verification.

The authors identify noise introduced with the traceroute data, such as
packet loss.  Other factors influence the results, including
inconsistent ROV enforcement within an AS, multihoming, traffic load
balancing, and path asymmetry.  Origin-only observation from SYN/ACK
responses to probes is the least stable method.  In this case, probes
towards Alex web sites may not all arrive at the same destination
instance.  Web sites may be distributed with content distribution
network, resulting in additional unfiltered noise.  If different probes
or responses traverse entirely different paths, filtering the results
will be required, but difficult since the only observation is in the
reply that returns to the source, which contains almost no path
information other than the TTL (hop limit in IPv6) without unreliable
options such as record route options.

Any claim of proof seems suspect unless the authors verified their
findings with network administrators.  They reason that an AS only sees
valid announcements and never invalid paths must be ROV enforcing.  They
do not make available the code or data to recreate this experiment, so
that this claim can be verified.  The results from these data plane
experiments verify even fewer ROV enforcing networks than reported in
\cite{reuter_towards_2018}.  False negatives with the data plane methods
outlined here are likely higher.  While the experiments are active and
some controlled, ambiguity is introduced when trying to evaluate ROV
enforcement when there are multiple intervening AS hops in a path.  It
is difficult if not impossible to deduce with certainty routing policy
that may exist between the different pairs of AS neighbors when
measuring from afar.

\subsection{Deployment Challenges}

As of this writing, depending on how one counts, deployed ROAs cover
approximately 10\% to 15\% of the IPv4 Internet.  While a direct
comparison cannot be made, signed DNSSEC zones see high levels of
participation at or near the top of the name space, but is much more
sparse below.  Nonetheless, many zones have been signed, but compared to
all possible zones, the coverage rate is about an order of magnitude
smaller than ROAs.  This may say more about the RPKI than DNSSEC.
DNSSEC is a much older technology, but there is also much more
individual DNS data to sign than routing data.  ROV adoption on the
other hand appears to still be in its infancy, but appears poised to see
steady growth and increased coverage of the IP address space.  While
deployment of the RPKI, ROAs, and ROV enforcement is encouraging,
barriers and difficulties remain.

\subsubsection{Political, Economic, and Social Limitations}

As reported in \cite{yoo_lowering_2018}, not all deployment challenges
are technical in nature.  For instance, ARIN, the North American RIR,
requires users of their portion of the RPKI to overcome strict legal
obstacles spelled out in their terms and conditions agreements.
Adoption in the ARIN region is believed to be artificially limited due to
these legal concerns.  Each RIR operates differently and these
differences are sometimes used to explain why ROA publication rates vary
dramatically from one RIR to another.  In fostering deployment of
securing routing, RIPE is statistically ahead by a significant amount.
In \cite{alex_band_ripe_2015}, it was demonstrated that a series of
focused interface improvements and ongoing constituent engagement helped
foster deployment of the RPKI in the RIPE community.

Perhaps the most cited deployment challenge with any new technology is
achieving critical mass when participation is its own reward, but where
there is little incentive for early adopters.  Publishing ROAs is
relatively easy, but there is little advantage in doing so if networks
are not performing ROV.  One suggestion by \cite{gilad_are_2017} is to
encourage ROV deployment in the core of the Internet, by the largest
ISPs.  Their evaluation argues that with just a modest amount of ROV
deployment in the core the value of ROAs increases are non-linear.

\subsubsection{Technical Hurdles}

New systems may also experience failures or begin with low quality data
as users and input are introduced.  Invalid ROAs pose a particular
problem as networks recoil from a technology if it is deemed not ready
for production use.  If an AS incorrectly populates the RPKI with an
incorrect ROA and other networks are ROV enforcing, the origin AS may
suddenly find themselves disconnected from the Internet.  This potential
problem has convinced some networks to only depreference invalid routes
rather than reject them outright.  Recent research has produced a number
of monitoring systems to help detect and foster deployment.  The Rov
Deployment Monitoring (RDM) system from the \cite{reuter_towards_2018}
team publishes a regular report on ROV deployment as detected by their
measurement methods.  The RDM provides a searchable interface to explore
the ROV status of observed networks.  The \cite{gilad_are_2017} team has
created the complimentary monitor known as ROAlert, which is concerned
with the signed data, or ROAs.  ROAlert highlights potential ROA usage
deployment problems such as address sub-allocation dependencies.

A particular problem for large ISPs has been twofold.  In publishing
ROAs, they must take care to avoid invalidating more specific routes
originating from customers.  For instance, if AS 49152 is allocated
192.0.2.0/24 and publishes a ROA for that specific prefix, but then
sub-allocates 192.0.2.128/25 to a customer 49153, the customer's
announcement will appear invalid until a customer-specific ROA is
published.  The other problem is when an ISP enforces ROV and ends up
rejecting routes that are invalid, but unintentionally so.  While this
may seem like a necessary trade-off, network users may often prefer to
connectivity over doing the right thing.  If a downstream network
becomes unreachable or a downstream network is unable to reach something
they care about due to ROV enforcement upstream, the downstream network
may decide to connect to another upstream network that doesn't burden
them with ROV enforcement.

One proposal to limiting the downward dependency conflicts is with the
introduction a new wildcard ROA.\cite{gilad_are_2017}  A wildcard ROA
would allow an AS to create another lower priority ROA for a aggregate
prefix without specifying a specific origin AS.  The wild card would
permit any other AS to announce smaller prefix from this aggregate,
without first having to create a specific ROA of their own.  This
syntactical problem with this solution is that the asID field in a ROA
is specified as an integer.  What value would represent a wildcard?  A
special value representing the wildcard function would need to be
assigned and then router implementations would need to know to recognize
this value.  The practical problem with this approach is that a wildcard
may now permit any origin AS to bypass ROV enforcement for the
associated prefix, including random malicious networks, posing a
potential hijack risk.  Notably, this solution seems to contradict the
concerns raised by the proposers for so-called \emph{loose} ROAs
this paper discusses below.

\subsubsection{Security Concerns}

At present, secure routing technology using the RPKI, ROAs, and ROV most
helpful in mitigating accidents and mistakes, but are unable to thwart
many intentionally malicious attacks.  an evil AS could modify the AS
path of a ROA-covered prefix to include a valid origin AS, but list
itself as a network on the path.  As long as the prefix announcement
validates according to a ROA, ROV will not detect this malicious route
announcement.  All else being equal, this increased path length may
limit its propagation, but it may be sufficient to hijack some traffic.
Alternatively, the malicious originator could announce a more specific
prefix if allowed by a ROA \emph{maxLength} value.

One of the few specified fields in a ROA, \emph{maxLength}, declares the
smallest prefix length that may be announced for the associated address
block.  For instance, imagine a ROA prefix is 192.0.2.0/24 but allows a
more specific announcement up to a /32, the prefix would be listed as
192.0.2.0/24 but the maxLength value would be 32.  The intent of the
maxLength attribute is to allow an origin AS some nimbleness to
disaggregate prefixes without have to create and distribute additional
ROAs.  However, \cite{gilad_are_2017} warn of a potential problems if
the steady state origin announcement is a typically the original prefix
length.  A malicious origin could come along and inject a more specific
prefix with the spoofed AS and pass ROV.  This technically does not
increase the risk of hijacking without ROV, but if the maxLength were
not used, it would invalidate more specific prefix hijacking attacks.
The routing community appears to be gravitating toward a stance of not
using the maxLength unless absolutely necessary.

In some scenarios, the deployment of ROV may lead to unexpected network
partitions or paths.  In partial ROV deployment scenarios, an ROV
enforcing network may or may not receive the protection it might expect.

Figure \ref{fig:collateral} recreates three possible scenarios described
in \cite{gilad_are_2017}.  \emph{(a)} depicts the ideal scenario in
partial ROV deployment.  Here AS 64510 performs ROV and protects 64511
from the invalid route originated by AS 64495.  In \emph{(b)}, AS 64510
does not perform ROV, but thd downstream AS 64511 does.  AS 64510
receives seemingly equal path options for the same prefix, but happens
to select and relay the invalid route to AS 64511.  Unless AS 64511 has
a default route through AS 64510 or another unseen path, it will be
disconnected from the valid origin AS 64495 entirely.  Disconnection may
be preferable to forwarding traffic toward the wrong network however.
Lastly, in \emph{(c)} a hijack occurs when  AS 64510 fails to perform
ROV and relays a covering valid /24 prefix and a covering invalid, but
more specific /24 prefix.   AS 64511 rejects the invalid more specific,
but it doesn't matter, because when it sends traffic to AS 64510
intended to reach the valid covering prefix the upstream AS  will follow
the more specific invalid path anyway.

\begin{figure*}
  \centering
    \includegraphics[width=1\textwidth]{collateral}
  \caption{Partial ROV adoption scenarios}
  \label{fig:collateral}
\end{figure*}

\section{Discussion}\label{sec:Discussion}

An Internet RPKI infrastructure to support secure routing is available
and actively being deployed.  Approximately 10\% of active AS networks
on the Internet have published ROAs in the RPKI.   Research measurement
studies attempting to uncover ROV enforcement deployment has uncovered
very little use of ROAs to secure routing thus far.  Nonetheless,
interest in ROV deployment appears to be on the verge of growth given
the recent rise in activity by operators and researchers.  Meanwhile
deployment challenges remain, stemming from hesitation by operators over
technical and organizational policy concerns and ending with technical
challenges involved in the roll out of most any new technology.
Researchers have produced a handful of alerting and monitoring tools to
help ease the transition to ROV and help validate proper ROA creation.

The earliest ROV measurement experiments utilized passive, uncontrolled
routing data from route collectors such as RouteViews.  While this
source of data can provide insights, passive, uncontrolled inferences
are prone to false positives and false negatives as a result of
ambiguities in local AS routing policy along the route path.  A
controlled, active measurement techniques decidedly improves the
robustness of ROV measurement research.  Observing controlled route
announcements at nearby vantage points can produce reliable discovery of
ROV enforcement on neighboring networks.

The research community has most recently proposed data plane techniques
to detect ROV adoption on network paths.  This effort can prove fruitful
with limited short AS paths, but ambiguity increases quickly after
the first AS hop.  Data plane methods currently provide limited utility
compared to the active, controlled experiment types, suggesting more
novel enhancements to this approach are needed to prove useful.

\section{Future Work}\label{sec:Future Work}

Achieving insight from running networks at a distance is sufficiently
difficult with existing measurement methods.  Operators may be reluctant
or unable to help provide direct access to their infrastructure for
research experiments so researchers must device innovate ways to uncover
RPKI usage and ROV enforcement.  Achieving additional visibility of ROV
enforcement from deployed routers is currently only available by local
administrators or through inference methods described above.  While some
inference methods such as controlled active measurement produce reliable
results, they offer a only a very partial view of the entire Internet
secure routing posture.  To better understand ROV enforcement as
deployed in the Internet, route collection systems should explicitly
mark routes with the local validity state.

Most modern routers maintain ROA validity state, but it remains hidden
within each router.  There are only three states, valid, invalid, or
unknown.  One approach would be to create a new, optional BGP, path
attribute that could carry the validity state with an update.  This
might also be useful by downstream networks who would prefer to rely on
the ROV mechanisms of their upstream provider, but it would be
particularly useful for route collection systems such as RouteViews,
significantly enhancing measurement studies.  It is no small feat to
alter the BGP specification however.  To evaluate this approach
a unique community tag could be used for this purpose initially.

The BGP Monitoring Protocol (BMP) is new capability of many
routers.\cite{scudder_bgp_2016} BMP exports routing table and peer
status to a BMP collector.  BMP is essence provides a remote, passive
view of BGP state to a collector.  BMP was designed to supplant and
improve upon more rudimentary route collection systems that are often
just \emph{screen scraping} the output of commands that display routing
information on a router.  BMP is ideally suited to augment routing data
collection with route validity state.  It would seem practically
feasible to extend BMP to include validity state in an extension or in a
protocol update.

\section{Conclusion}\label{sec:Conclusion}

This paper considered a primary set of related papers on the deployment
of secure routing using the RPKI and with a particular on ROV
measurement methodologies on the Internet.  A background of routing
technologies and features that have traditionally be used to secure
routing was covered in order to level set a discussion involving the
modern RPKI-based.  The design of the RPKI, ROAs, and ROV appear to have
been influenced by lessons from previous technologies or secure routing
proposals as deployment appears on a steady upward trajectory.
Measuring just how deeply and widely ROV enforcement is occurring however
is a challenge due to the inaccessible local policy decisions at each AS
from would-be outside researchers.  Nonetheless a handful of techniques
have been devised to infer ROV deployment and enforcement activity with
varying levels of reliability and reach.  Controlled active measurement
techniques are especially promising.  To increase the breadth of ROV
enforcement detection however, new techniques or sources of route
validity will be required.

\begin{acks}

John is grateful for the networking position he holds at DePaul
University, which facilitates the ability to conduct meaningful
experiments such as those in ARIN's operational test and evaluation
environment (OTE) used to better understand the technology presented in
this paper.  John is also indebted to Patrick W.  Gilmore, who
generously made his time available to answer questions about the 1997
\emph{7007 Incident}.

\end{acks}

\bibliography{wcp}

\end{document}
