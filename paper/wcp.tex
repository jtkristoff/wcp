\documentclass[sigconf]{acmart}

% https://tex.stackexchange.com/questions/346292/how-to-remove-conference-information-from-the-acm-2017-sigconf-template
\settopmatter{printacmref=false} % Removes citation information below abstract
\renewcommand\footnotetextcopyrightpermission[1]{} % removes footnote with conference information in first column
\pagestyle{plain} % removes running headers

\bibliographystyle{ACM-Reference-Format}

\usepackage{graphicx}
\usepackage{listings}

\begin{document}

% use protected (~) whitespace to control title on second line
\title{The RPKI and Route Origin Validation: Advances~in~Deployment~and~Measurement}
\subtitle{PhD Qualifier Examination - Critical Review}
\author{John Kristoff}
\affiliation{\department{Computer Science}\institution{University of Illinois at Chicago}}
\email{jkrist3@uic.edu}

\begin{abstract}

Routing leaks and hijacks have plagued the Internet ecosystem since the
late 1990's.  These events, often unintentional, are recurring reminders
of the risk the Internet infrastructure is subject to.  The Resource
Public Key Infrastructure (RPKI) has arisen as an important and
necessary component to facilitate secure Internet routing.  Many network
operators are now publishing digitally signed certificates of address
prefixes and their associated route origins into the RPKI.  A smaller,
but burgeoning community of operators have begun deploying route origin
validation capabilities to limit the propagation of invalid routing
information in disagreement with the RPKI.  Now researchers are building
systems and techniques to evaluate how the RPKI is being populated and
used, drawing comparisons to other Internet infrastructure security
services such as the extensions to the domain name system.  Private and
localized validation policy decisions pose measurement challenges for
would-be secure routing surveyors however.  Operator polling, passive
observation of route updates, controlled route announcement experiments,
and data plane analysis are among the common techniques devised to
estimate deployment today.  When operators and route collection tools
incorporate route validation state in data collection systems,
researchers will be able to enhance their methods further still.

\end{abstract}

\maketitle

\section{Introduction}\label{sec:Introduction}

The Internet is a network of networks.  Each network that forms the
larger whole is commonly represented and referred to as an autonomous
system (AS).  To form a loop-free connected graph, an AS typically uses
the border gateway protocol (BGP), the Internet routing standard defined
in IETF RFC 4271 to connect to one or more other autonomous
systems.\cite{rekhter_border_2006}  AS-to-AS routers exchange and relay
reachability information using BGP update messages, which include one or
more address prefixes, an associated origin AS identifier, and other
attributes as appropriate.

BGP is unique among most common Internet standard routing protocols due
to the varied and widespread use of local policy configuration options
available at each router node and within an AS.  Each AS therefore may
form different views of the larger Internet topology by accepting,
rejecting, relaying, or altering routing information.  That is, while a
loop-free topology remains a critical feature of BGP, the shortest path
to any destination may be a secondary goal depending on local
administrator policy.  For example, one network may choose to reject or
withhold a route update containing a particular AS deemed untrustworty.
To illustrate another common scenario, an AS may devalue routing
information in order to prefer one path over another.  These types of
local policy decisions affect local path computations, but can also
impose policy onto neighboring networks and beyond, narrowing the path
selection decisions further downstream.

In addition to the added complexity local policy configurations imply,
the data conveyed in routing update messages between BGP speakers come
with no express means of authenticity.  This means that by default, as
long as the BGP message is well formed, a routing update with invalid
path data may find its way into the path computation process, wreaking
havoc and corrupting the computed topology of one or more autonomous
systems.  How does a BGP receiver evaluate routing information when an
autonomous neighbor is untrustworthy either due to accident or malicious
intent?  Approaches to protection from invalid routing data have varied,
haphazzadly dealt with in each autonomous system as a local policy
decision.  Consequently, overall Internet routing infrastructure
robustness against invalid routing data varies widely from AS to AS.
While a variety of solutions have been deployed, the Internet has lacked
a comprehensive, standard approach to authenticating routing data
distributed through BGP.  Unfortunately, the threats of invalid routing
data are not hypothetical.

In 1997, a unique set of circumstances coalesced at just the right place
and time to cause a major disruption in the Internet routing
system.\cite{barret_routing_1997}  A small Internet service provider
(ISP) both disaggregated and regenerated a large number of Internet
address prefixes back into the Internet, making it appear as if the
address prefixes in these routing updates all originated from the small
ISP.  These numerous and errorneous updates cascaded to many other
networks.  To many Internet router configurations the updates were
accepted and used to recompute paths.  Instead of traffic following the
best path during this event, many of intended destinations covered by
these updates became unreachable as routers started forwarding traffic
towards the wrong ISP.

Just over a decade later, another widely observed routing anomaly
impacted the popular YouTube streaming video
service.\cite{brown_pakistan_2008}  Pakistan Telecom, intending to block
access to YouTube from within its network accidentally leaked a specific
address prefix covering a portion of YouTube's address space in a route
update to an upstream provider.  This in turn propagated as a preferred
path to many other networks.  The proliferation of this update sent many
of the world's YouTube users towards Pakistan Telecom, which appeared in
the routing system as the best path, but effectively led to an Internet
version of a black hole.

The aforementioned network infrastructure mishaps, commonly referred to
as a \emph{hijack}, are among the most popular in the history of
Internet routing.  However, events such these, even when less well
known, recur with surprising frequency.  Often these events can be
classified as accidents, but there is evidence many anomalies have been
intentionally malicious.\cite{madory_bgp_2018}  Whether the events are
benign or costly, inadvertent or deliberate, mechanisms to limit their
occurrence have been an ongoing area of research, experimentation, and
development.

This paper evaluates the deployment and measurement efforts of a new
approach to hardening the Internet routing system against illegimate
routing updates.  The rest of the paper is organized as follows: In \S
\ref{sec:Background} the history of secure routing is divided into two
distinct eras.  The first era summarizes popular techniques and tools
network operators have traditionally deployed to limit the spread of
invalid or unwanted routing information.  The second era includes the
recent design of the resource public key infrastructure (RPKI), route
origin validation (ROV), and related standards efforts that form the
basis for the bulk of this paper.  In \S \ref{sec:Methodologies},
approaches to measuring the deployment of the RPKI and ROV from three
recent research papers that represent the state of the art are
summarized.  In \S \ref{sec:Discussion}, this paper offers fresh
perspective and criticism of the prior research under evaluation.  In \S
\ref{sec:Future Work}, an argument is made to include and leverage the
route validation state of ROV-enabled networks in route monitoring
systems to encourage deployment and enhance measurement.  The paper
concludes in \S \ref{sec:Conclusion}.

\section{Background}\label{sec:Background}

Unique to BGP is the rich feature set the protocol and router
implementations make available.  An administrator can easily suppress,
create, or alter BGP route update messages.  These features provide
enormous flexibility and many networks make use of these capabilities
within their network in a variety of innovative ways.  The ease and
frequency with which these features can be used often lead to complexity
and uncertainty.  Indeed, across different autonomous systems,
differences in \emph{BGP policy} can present very different views of the
global Internet routing system.

While a complete tutorial and history of secure Internet routing is
beyond the scope of this paper, a brief review of popular approaches and
technologies relevant to understanding the RPKI and ROV is provided.
The reader is encouraged to review the references provided for
additional detail.

\subsection{Route Filters and Prefix Limits}

Two BGP capabilities frequently used to limit undesirable or unexpected
routing updates are route filters and prefix limits.  A route filter is
list of one or more rules that evaluates a received announcement in a
route update and decides whether to accept, reject, or alter the
announcement for local route computation.  A prefix limit is a
threshold, an expected upper limit of prefixes a BGP neighbor is
expected to announce to the receiving router.  A breach of a prefix
limit may trigger an alert or shut down the BGP peering session.

Network providers commonly use route filters to reject route
announcements they never expect to see.  For instance, if AS
\emph{49152} is the sole originator of IPv4 address prefix
\texttt{192.0.2.0/24}, an appropriate route filter might reject any
announcement for \texttt{192.0.2.0/24} or smaller overlapping prefix
from other networks.  See Listing~\ref{lst:route filter} for an example
BGP policy implementing a route filter to reject any route announcements
for any address space covered by IETF RFC 1918
prefixes.\cite{moskowitz_address_1996}

\begin{lstlisting}[basicstyle=\footnotesize\ttfamily,caption={Example Junos route filter},label={lst:route filter}]
policy-options {
    prefix-list rfc1918 {
        10.0.0.0/8;                         
        172.16.0.0/12;                      
        192.168.0.0/16;                     
    }
    policy-statement sanitize-bgp {
        term rfc1918 {
            from {
                prefix-list-filter rfc1918 orlonger;
            }
            then reject;
        }
    }
}
\end{lstlisting}

Route filters can range from the relatively simple to thoroughly
complex.  Most router implementations provide an ability to peek into
almost all parts of the routing announcement and pass judgment in myriad
ways.  A creative operator may be able to prevent many invalid routing
updates using an elaborate set of route filters, but achieving complete
protection from all possible invalid routes would be a Herculean, if not
impossible task.  Nevertheless, route filters along with prefix limits
have been two of the most common and widely deployed capabilities to
limit invalid route propagtion on the Internet
today.\cite{durand_bgp_2015}

\subsection{Internet Routing Registries}

Since the early years of BGP deployment, the Internet community
identified the need for tools and a set of best practices to support the
sprawling routing infrastructure.  One of the early developments to
support these early challenges was the design and creation of Internet
Routing Registries (IRRs).\cite{bates_representation_1995}  The IRR
system was an outgrowth of the desire to express an autonomous system's
routing policy, primarily for troubleshooting and documentation
purposes.  Tools to help automate verifying routing data against IRR
data soon appeared.  In fact, when the \emph{AS 7007
Incident}\footnote{This is the name given to the routing event in 1997
described in this paper above.  The 7007 refers to the autonomous system
number assigned to the small ISP originating the invalid routes.}
occurred a number of operators suggested that the use of a routing
registry could have prevented the problem.

IRRs and IRR tools have been adopted by a number of networks, but the
data has been of varying completeness and
quality.\cite{khan_comparative_2013}  Perhaps the IRR system was ahead
of its time, or maybe it hasn't struck the right balance of cost,
consistency, ease of use, and utility for it to have become universally
deployed as a routing security mechanism.  It would take a number of
years and a few proposals before a system focused on solely on routing
security would take hold.

\subsection{S-BGP and soBGP}

At the turn of the 21st century, two extensions to BGP, S-BGP and soBGP,
were similar attempts to bring cryptographic authentication to routing
data.\cite{kent_secure_2000}\cite{white_securing_2003}  Both approaches
specify a public key infrastructure (PKI) and both modify BGP to
incorporate authentication into the routing protocol.  Neither approach
gained sufficient traction to become deployed systems however.
Nonetheless, their use of a PKI foreshadowed a system to come that would
capture the imagination of operators and soon demonstrate promising
levels of real-world deployment.

\subsection{The RPKI}

\subsection{ROAs}

\subsection{ROV}

Addresses/Prefixes.  ASNs.  RIRs.  Routing.  Routing security.  History
and time line to RPKI.

The Resource Public Key Infrastructure (RPKI), an Internet
routing-specific PKI, anchored at the five regional Internet registries
(RIRs) is the starting point for a repository of cryptographically
signed route object authorizations (ROAs).  ROAs

\section{Methodologies}\label{sec:Methodologies}

Hmm.

\section{Discussion}\label{sec:Discussion}

Hah.

\section{Future Work}\label{sec:Future Work}

Baz.

\section{Conclusion}\label{sec:Conclusion}

Baf.

\begin{acks}

John is grateful for the networking position he holds at DePaul
University, which facilitates the ability to conduct meaningful
experiments such as those in ARIN's operational test and evaluation
environment (OTE).  John is also indebted to Patrick W.  Gilmore, who
generously made his time available to answer questions about the 1997
routing event commonly known as the \emph{7007 Incident}.

\end{acks}

\bibliography{wcp}

\end{document}
